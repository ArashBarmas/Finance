\documentclass[]{article}
\usepackage{lmodern}
\usepackage{amssymb,amsmath}
\usepackage{ifxetex,ifluatex}
\usepackage{fixltx2e} % provides \textsubscript
\ifnum 0\ifxetex 1\fi\ifluatex 1\fi=0 % if pdftex
  \usepackage[T1]{fontenc}
  \usepackage[utf8]{inputenc}
\else % if luatex or xelatex
  \ifxetex
    \usepackage{mathspec}
  \else
    \usepackage{fontspec}
  \fi
  \defaultfontfeatures{Ligatures=TeX,Scale=MatchLowercase}
\fi
% use upquote if available, for straight quotes in verbatim environments
\IfFileExists{upquote.sty}{\usepackage{upquote}}{}
% use microtype if available
\IfFileExists{microtype.sty}{%
\usepackage{microtype}
\UseMicrotypeSet[protrusion]{basicmath} % disable protrusion for tt fonts
}{}
\usepackage[margin=1in]{geometry}
\usepackage{hyperref}
\hypersetup{unicode=true,
            pdftitle={Econ\_HW4},
            pdfauthor={Arash Barmas},
            pdfborder={0 0 0},
            breaklinks=true}
\urlstyle{same}  % don't use monospace font for urls
\usepackage{color}
\usepackage{fancyvrb}
\newcommand{\VerbBar}{|}
\newcommand{\VERB}{\Verb[commandchars=\\\{\}]}
\DefineVerbatimEnvironment{Highlighting}{Verbatim}{commandchars=\\\{\}}
% Add ',fontsize=\small' for more characters per line
\usepackage{framed}
\definecolor{shadecolor}{RGB}{248,248,248}
\newenvironment{Shaded}{\begin{snugshade}}{\end{snugshade}}
\newcommand{\KeywordTok}[1]{\textcolor[rgb]{0.13,0.29,0.53}{\textbf{#1}}}
\newcommand{\DataTypeTok}[1]{\textcolor[rgb]{0.13,0.29,0.53}{#1}}
\newcommand{\DecValTok}[1]{\textcolor[rgb]{0.00,0.00,0.81}{#1}}
\newcommand{\BaseNTok}[1]{\textcolor[rgb]{0.00,0.00,0.81}{#1}}
\newcommand{\FloatTok}[1]{\textcolor[rgb]{0.00,0.00,0.81}{#1}}
\newcommand{\ConstantTok}[1]{\textcolor[rgb]{0.00,0.00,0.00}{#1}}
\newcommand{\CharTok}[1]{\textcolor[rgb]{0.31,0.60,0.02}{#1}}
\newcommand{\SpecialCharTok}[1]{\textcolor[rgb]{0.00,0.00,0.00}{#1}}
\newcommand{\StringTok}[1]{\textcolor[rgb]{0.31,0.60,0.02}{#1}}
\newcommand{\VerbatimStringTok}[1]{\textcolor[rgb]{0.31,0.60,0.02}{#1}}
\newcommand{\SpecialStringTok}[1]{\textcolor[rgb]{0.31,0.60,0.02}{#1}}
\newcommand{\ImportTok}[1]{#1}
\newcommand{\CommentTok}[1]{\textcolor[rgb]{0.56,0.35,0.01}{\textit{#1}}}
\newcommand{\DocumentationTok}[1]{\textcolor[rgb]{0.56,0.35,0.01}{\textbf{\textit{#1}}}}
\newcommand{\AnnotationTok}[1]{\textcolor[rgb]{0.56,0.35,0.01}{\textbf{\textit{#1}}}}
\newcommand{\CommentVarTok}[1]{\textcolor[rgb]{0.56,0.35,0.01}{\textbf{\textit{#1}}}}
\newcommand{\OtherTok}[1]{\textcolor[rgb]{0.56,0.35,0.01}{#1}}
\newcommand{\FunctionTok}[1]{\textcolor[rgb]{0.00,0.00,0.00}{#1}}
\newcommand{\VariableTok}[1]{\textcolor[rgb]{0.00,0.00,0.00}{#1}}
\newcommand{\ControlFlowTok}[1]{\textcolor[rgb]{0.13,0.29,0.53}{\textbf{#1}}}
\newcommand{\OperatorTok}[1]{\textcolor[rgb]{0.81,0.36,0.00}{\textbf{#1}}}
\newcommand{\BuiltInTok}[1]{#1}
\newcommand{\ExtensionTok}[1]{#1}
\newcommand{\PreprocessorTok}[1]{\textcolor[rgb]{0.56,0.35,0.01}{\textit{#1}}}
\newcommand{\AttributeTok}[1]{\textcolor[rgb]{0.77,0.63,0.00}{#1}}
\newcommand{\RegionMarkerTok}[1]{#1}
\newcommand{\InformationTok}[1]{\textcolor[rgb]{0.56,0.35,0.01}{\textbf{\textit{#1}}}}
\newcommand{\WarningTok}[1]{\textcolor[rgb]{0.56,0.35,0.01}{\textbf{\textit{#1}}}}
\newcommand{\AlertTok}[1]{\textcolor[rgb]{0.94,0.16,0.16}{#1}}
\newcommand{\ErrorTok}[1]{\textcolor[rgb]{0.64,0.00,0.00}{\textbf{#1}}}
\newcommand{\NormalTok}[1]{#1}
\usepackage{graphicx,grffile}
\makeatletter
\def\maxwidth{\ifdim\Gin@nat@width>\linewidth\linewidth\else\Gin@nat@width\fi}
\def\maxheight{\ifdim\Gin@nat@height>\textheight\textheight\else\Gin@nat@height\fi}
\makeatother
% Scale images if necessary, so that they will not overflow the page
% margins by default, and it is still possible to overwrite the defaults
% using explicit options in \includegraphics[width, height, ...]{}
\setkeys{Gin}{width=\maxwidth,height=\maxheight,keepaspectratio}
\IfFileExists{parskip.sty}{%
\usepackage{parskip}
}{% else
\setlength{\parindent}{0pt}
\setlength{\parskip}{6pt plus 2pt minus 1pt}
}
\setlength{\emergencystretch}{3em}  % prevent overfull lines
\providecommand{\tightlist}{%
  \setlength{\itemsep}{0pt}\setlength{\parskip}{0pt}}
\setcounter{secnumdepth}{0}
% Redefines (sub)paragraphs to behave more like sections
\ifx\paragraph\undefined\else
\let\oldparagraph\paragraph
\renewcommand{\paragraph}[1]{\oldparagraph{#1}\mbox{}}
\fi
\ifx\subparagraph\undefined\else
\let\oldsubparagraph\subparagraph
\renewcommand{\subparagraph}[1]{\oldsubparagraph{#1}\mbox{}}
\fi

%%% Use protect on footnotes to avoid problems with footnotes in titles
\let\rmarkdownfootnote\footnote%
\def\footnote{\protect\rmarkdownfootnote}

%%% Change title format to be more compact
\usepackage{titling}

% Create subtitle command for use in maketitle
\newcommand{\subtitle}[1]{
  \posttitle{
    \begin{center}\large#1\end{center}
    }
}

\setlength{\droptitle}{-2em}

  \title{Econ\_HW4}
    \pretitle{\vspace{\droptitle}\centering\huge}
  \posttitle{\par}
    \author{Arash Barmas}
    \preauthor{\centering\large\emph}
  \postauthor{\par}
      \predate{\centering\large\emph}
  \postdate{\par}
    \date{2/28/2019}


\begin{document}
\maketitle

R Exercise A Question 1 We have an MA(1) model Simulate and plot 250
obser- vations of the MA(1)

1 a)

\begin{Shaded}
\begin{Highlighting}[]
\NormalTok{theta =}\StringTok{ }\FloatTok{0.5}
\NormalTok{mu    =}\StringTok{ }\FloatTok{0.05}
\NormalTok{ma1.model.}\DecValTok{5}\NormalTok{ =}\StringTok{ }\KeywordTok{list}\NormalTok{(}\DataTypeTok{ma=}\NormalTok{theta)}

\KeywordTok{set.seed}\NormalTok{(}\DecValTok{123}\NormalTok{)}
\NormalTok{ma1.sim.}\DecValTok{5}\NormalTok{ =}\StringTok{ }\NormalTok{mu }\OperatorTok{+}\StringTok{ }\KeywordTok{arima.sim}\NormalTok{(}\DataTypeTok{model=}\NormalTok{ma1.model.}\DecValTok{5}\NormalTok{, }\DataTypeTok{n=}\DecValTok{250}\NormalTok{,}
                         \DataTypeTok{innov=}\KeywordTok{rnorm}\NormalTok{(}\DataTypeTok{n=}\DecValTok{250}\NormalTok{, }\DataTypeTok{mean=}\DecValTok{0}\NormalTok{, }\DataTypeTok{sd=}\FloatTok{0.1}\NormalTok{))}

\NormalTok{acf.ma1.model.}\DecValTok{5}\NormalTok{ =}\StringTok{ }\KeywordTok{ARMAacf}\NormalTok{(}\DataTypeTok{ma=}\NormalTok{theta, }\DataTypeTok{lag.max=}\DecValTok{10}\NormalTok{)}
\KeywordTok{par}\NormalTok{(}\DataTypeTok{mfrow=}\KeywordTok{c}\NormalTok{(}\DecValTok{3}\NormalTok{,}\DecValTok{1}\NormalTok{))}
    \KeywordTok{plot}\NormalTok{(ma1.sim.}\DecValTok{5}\NormalTok{, }\DataTypeTok{main=}\StringTok{"MA(1) Process: mu=0.05, theta=0.5"}\NormalTok{,}\DataTypeTok{xlab=}\StringTok{"time"}\NormalTok{,}\DataTypeTok{ylab=}\StringTok{"y(t)"}\NormalTok{)}
    \KeywordTok{abline}\NormalTok{(}\DataTypeTok{h=}\DecValTok{0}\NormalTok{)}
    \KeywordTok{plot}\NormalTok{(}\DecValTok{1}\OperatorTok{:}\DecValTok{11}\NormalTok{, acf.ma1.model.}\DecValTok{5}\NormalTok{, }\DataTypeTok{type=}\StringTok{"h"}\NormalTok{, }\DataTypeTok{col=}\StringTok{"blue"}\NormalTok{, }\DataTypeTok{main=}\StringTok{"Theoretical ACF"}\NormalTok{)}
\NormalTok{    tmp=}\KeywordTok{acf}\NormalTok{(ma1.sim.}\DecValTok{5}\NormalTok{, }\DataTypeTok{lag.max=}\DecValTok{10}\NormalTok{, }\DataTypeTok{main=}\StringTok{"Sample ACF"}\NormalTok{)}
\end{Highlighting}
\end{Shaded}

\includegraphics{Econ_HW4_files/figure-latex/unnamed-chunk-1-1.pdf}

\begin{Shaded}
\begin{Highlighting}[]
\NormalTok{theta =}\StringTok{ }\FloatTok{0.9}
\NormalTok{mu    =}\StringTok{ }\FloatTok{0.05}
\NormalTok{ma1.model.}\DecValTok{5}\NormalTok{ =}\StringTok{ }\KeywordTok{list}\NormalTok{(}\DataTypeTok{ma=}\NormalTok{theta)}

\KeywordTok{set.seed}\NormalTok{(}\DecValTok{123}\NormalTok{)}
\NormalTok{ma1.sim.}\DecValTok{5}\NormalTok{ =}\StringTok{ }\NormalTok{mu }\OperatorTok{+}\StringTok{ }\KeywordTok{arima.sim}\NormalTok{(}\DataTypeTok{model=}\NormalTok{ma1.model.}\DecValTok{5}\NormalTok{, }\DataTypeTok{n=}\DecValTok{250}\NormalTok{,}
                         \DataTypeTok{innov=}\KeywordTok{rnorm}\NormalTok{(}\DataTypeTok{n=}\DecValTok{250}\NormalTok{, }\DataTypeTok{mean=}\DecValTok{0}\NormalTok{, }\DataTypeTok{sd=}\FloatTok{0.1}\NormalTok{))}

\NormalTok{acf.ma1.model.}\DecValTok{5}\NormalTok{ =}\StringTok{ }\KeywordTok{ARMAacf}\NormalTok{(}\DataTypeTok{ma=}\NormalTok{theta, }\DataTypeTok{lag.max=}\DecValTok{10}\NormalTok{)}
\KeywordTok{par}\NormalTok{(}\DataTypeTok{mfrow=}\KeywordTok{c}\NormalTok{(}\DecValTok{3}\NormalTok{,}\DecValTok{1}\NormalTok{))}
    \KeywordTok{plot}\NormalTok{(ma1.sim.}\DecValTok{5}\NormalTok{, }\DataTypeTok{main=}\StringTok{"MA(1) Process: mu=0.05, theta=0.9"}\NormalTok{,}\DataTypeTok{xlab=}\StringTok{"time"}\NormalTok{,}\DataTypeTok{ylab=}\StringTok{"y(t)"}\NormalTok{)}
    \KeywordTok{abline}\NormalTok{(}\DataTypeTok{h=}\DecValTok{0}\NormalTok{)}
    \KeywordTok{plot}\NormalTok{(}\DecValTok{1}\OperatorTok{:}\DecValTok{11}\NormalTok{, acf.ma1.model.}\DecValTok{5}\NormalTok{, }\DataTypeTok{type=}\StringTok{"h"}\NormalTok{, }\DataTypeTok{col=}\StringTok{"blue"}\NormalTok{, }\DataTypeTok{main=}\StringTok{"Theoretical ACF"}\NormalTok{)}
\NormalTok{    tmp=}\KeywordTok{acf}\NormalTok{(ma1.sim.}\DecValTok{5}\NormalTok{, }\DataTypeTok{lag.max=}\DecValTok{10}\NormalTok{, }\DataTypeTok{main=}\StringTok{"Sample ACF"}\NormalTok{)}
\end{Highlighting}
\end{Shaded}

\includegraphics{Econ_HW4_files/figure-latex/unnamed-chunk-2-1.pdf}

\begin{Shaded}
\begin{Highlighting}[]
\NormalTok{theta =}\StringTok{ }\OperatorTok{-}\FloatTok{0.9}
\NormalTok{mu    =}\StringTok{ }\FloatTok{0.05}
\NormalTok{ma1.model.}\DecValTok{5}\NormalTok{ =}\StringTok{ }\KeywordTok{list}\NormalTok{(}\DataTypeTok{ma=}\NormalTok{theta)}

\KeywordTok{set.seed}\NormalTok{(}\DecValTok{123}\NormalTok{)}
\NormalTok{ma1.sim.}\DecValTok{5}\NormalTok{ =}\StringTok{ }\NormalTok{mu }\OperatorTok{+}\StringTok{ }\KeywordTok{arima.sim}\NormalTok{(}\DataTypeTok{model=}\NormalTok{ma1.model.}\DecValTok{5}\NormalTok{, }\DataTypeTok{n=}\DecValTok{250}\NormalTok{,}
                         \DataTypeTok{innov=}\KeywordTok{rnorm}\NormalTok{(}\DataTypeTok{n=}\DecValTok{250}\NormalTok{, }\DataTypeTok{mean=}\DecValTok{0}\NormalTok{, }\DataTypeTok{sd=}\FloatTok{0.1}\NormalTok{))}

\NormalTok{acf.ma1.model.}\DecValTok{5}\NormalTok{ =}\StringTok{ }\KeywordTok{ARMAacf}\NormalTok{(}\DataTypeTok{ma=}\NormalTok{theta, }\DataTypeTok{lag.max=}\DecValTok{10}\NormalTok{)}
\KeywordTok{par}\NormalTok{(}\DataTypeTok{mfrow=}\KeywordTok{c}\NormalTok{(}\DecValTok{3}\NormalTok{,}\DecValTok{1}\NormalTok{))}
    \KeywordTok{plot}\NormalTok{(ma1.sim.}\DecValTok{5}\NormalTok{, }\DataTypeTok{main=}\StringTok{"MA(1) Process: mu=0.05, theta=-0.9"}\NormalTok{,}\DataTypeTok{xlab=}\StringTok{"time"}\NormalTok{,}\DataTypeTok{ylab=}\StringTok{"y(t)"}\NormalTok{)}
    \KeywordTok{abline}\NormalTok{(}\DataTypeTok{h=}\DecValTok{0}\NormalTok{)}
    \KeywordTok{plot}\NormalTok{(}\DecValTok{1}\OperatorTok{:}\DecValTok{11}\NormalTok{, acf.ma1.model.}\DecValTok{5}\NormalTok{, }\DataTypeTok{type=}\StringTok{"h"}\NormalTok{, }\DataTypeTok{col=}\StringTok{"blue"}\NormalTok{, }\DataTypeTok{main=}\StringTok{"Theoretical ACF"}\NormalTok{)}
\NormalTok{    tmp=}\KeywordTok{acf}\NormalTok{(ma1.sim.}\DecValTok{5}\NormalTok{, }\DataTypeTok{lag.max=}\DecValTok{10}\NormalTok{, }\DataTypeTok{main=}\StringTok{"Sample ACF"}\NormalTok{)}
\end{Highlighting}
\end{Shaded}

\includegraphics{Econ_HW4_files/figure-latex/unnamed-chunk-3-1.pdf}

1 b) As we know our model is MA(1), so Y(t) is a linear combination of
mu, e(t), and e(t-1). Also, the sign of the correlations of two RV with
lag 1 (Y(t) and Y(t-1)) is determined by theta. By increasing theta, the
first lag on the ACF didn't change much.

Question 2

2 a)

\begin{Shaded}
\begin{Highlighting}[]
\NormalTok{phi =}\StringTok{ }\DecValTok{0} 
\NormalTok{mu  =}\StringTok{ }\FloatTok{0.05}
\NormalTok{ar1.model.}\DecValTok{5}\NormalTok{ =}\StringTok{ }\KeywordTok{list}\NormalTok{(}\DataTypeTok{ar=}\NormalTok{phi)}

\KeywordTok{set.seed}\NormalTok{(}\DecValTok{123}\NormalTok{)}
\NormalTok{ar1.sim.}\DecValTok{5}\NormalTok{ =}\StringTok{ }\KeywordTok{suppressWarnings}\NormalTok{(mu}\OperatorTok{*}\NormalTok{(}\DecValTok{1}\OperatorTok{-}\NormalTok{phi) }\OperatorTok{+}\StringTok{ }\KeywordTok{arima.sim}\NormalTok{(}\DataTypeTok{model=}\NormalTok{ar1.model.}\DecValTok{5}\NormalTok{, }\DataTypeTok{n =} \DecValTok{250}\NormalTok{,}
                             \DataTypeTok{innov=}\KeywordTok{rnorm}\NormalTok{(}\DataTypeTok{n=}\DecValTok{250}\NormalTok{, }\DataTypeTok{mean=}\DecValTok{0}\NormalTok{, }\DataTypeTok{sd=}\FloatTok{0.1}\NormalTok{)))}
\NormalTok{acf.ar1.model.}\DecValTok{5}\NormalTok{ =}\StringTok{ }\KeywordTok{ARMAacf}\NormalTok{(}\DataTypeTok{ar=}\NormalTok{phi, }\DataTypeTok{lag.max=}\DecValTok{10}\NormalTok{)}

\KeywordTok{par}\NormalTok{(}\DataTypeTok{mfrow=}\KeywordTok{c}\NormalTok{(}\DecValTok{3}\NormalTok{,}\DecValTok{1}\NormalTok{))}
    \KeywordTok{plot}\NormalTok{(ar1.sim.}\DecValTok{5}\NormalTok{, }\DataTypeTok{main=}\StringTok{"AR(1) Process: mu=0.05, phi=0"}\NormalTok{,}
           \DataTypeTok{xlab=}\StringTok{"time"}\NormalTok{,}\DataTypeTok{ylab=}\StringTok{"y(t)"}\NormalTok{)}
    \KeywordTok{abline}\NormalTok{(}\DataTypeTok{h=}\DecValTok{0}\NormalTok{)}
    \KeywordTok{plot}\NormalTok{(}\DecValTok{1}\OperatorTok{:}\DecValTok{11}\NormalTok{, acf.ar1.model.}\DecValTok{5}\NormalTok{, }\DataTypeTok{type=}\StringTok{"h"}\NormalTok{, }\DataTypeTok{col=}\StringTok{"blue"}\NormalTok{, }\DataTypeTok{main=}\StringTok{"Theoretical ACF"}\NormalTok{)}
\NormalTok{    tmp=}\KeywordTok{acf}\NormalTok{(ar1.sim.}\DecValTok{5}\NormalTok{, }\DataTypeTok{lag.max=}\DecValTok{10}\NormalTok{, }\DataTypeTok{main=}\StringTok{"Sample ACF"}\NormalTok{)}
\end{Highlighting}
\end{Shaded}

\includegraphics{Econ_HW4_files/figure-latex/unnamed-chunk-4-1.pdf}

\begin{Shaded}
\begin{Highlighting}[]
\NormalTok{phi =}\StringTok{ }\FloatTok{0.5}
\NormalTok{mu  =}\StringTok{ }\FloatTok{0.05}
\NormalTok{ar1.model.}\DecValTok{5}\NormalTok{ =}\StringTok{ }\KeywordTok{list}\NormalTok{(}\DataTypeTok{ar=}\NormalTok{phi)}

\KeywordTok{set.seed}\NormalTok{(}\DecValTok{123}\NormalTok{)}
\NormalTok{ar1.sim.}\DecValTok{5}\NormalTok{ =}\StringTok{ }\NormalTok{mu}\OperatorTok{*}\NormalTok{(}\DecValTok{1}\OperatorTok{-}\NormalTok{phi) }\OperatorTok{+}\StringTok{ }\KeywordTok{arima.sim}\NormalTok{(}\DataTypeTok{model=}\NormalTok{ar1.model.}\DecValTok{5}\NormalTok{, }\DataTypeTok{n =} \DecValTok{250}\NormalTok{,}
                             \DataTypeTok{innov=}\KeywordTok{rnorm}\NormalTok{(}\DataTypeTok{n=}\DecValTok{250}\NormalTok{, }\DataTypeTok{mean=}\DecValTok{0}\NormalTok{, }\DataTypeTok{sd=}\FloatTok{0.1}\NormalTok{))}
\NormalTok{acf.ar1.model.}\DecValTok{5}\NormalTok{ =}\StringTok{ }\KeywordTok{ARMAacf}\NormalTok{(}\DataTypeTok{ar=}\NormalTok{phi, }\DataTypeTok{lag.max=}\DecValTok{10}\NormalTok{)}

\KeywordTok{par}\NormalTok{(}\DataTypeTok{mfrow=}\KeywordTok{c}\NormalTok{(}\DecValTok{3}\NormalTok{,}\DecValTok{1}\NormalTok{))}
    \KeywordTok{plot}\NormalTok{(ar1.sim.}\DecValTok{5}\NormalTok{, }\DataTypeTok{main=}\StringTok{"AR(1) Process: mu=0.05, phi=0.5"}\NormalTok{,}
           \DataTypeTok{xlab=}\StringTok{"time"}\NormalTok{,}\DataTypeTok{ylab=}\StringTok{"y(t)"}\NormalTok{)}
    \KeywordTok{abline}\NormalTok{(}\DataTypeTok{h=}\DecValTok{0}\NormalTok{)}
    \KeywordTok{plot}\NormalTok{(}\DecValTok{1}\OperatorTok{:}\DecValTok{11}\NormalTok{, acf.ar1.model.}\DecValTok{5}\NormalTok{, }\DataTypeTok{type=}\StringTok{"h"}\NormalTok{, }\DataTypeTok{col=}\StringTok{"blue"}\NormalTok{, }\DataTypeTok{main=}\StringTok{"Theoretical ACF"}\NormalTok{)}
\NormalTok{    tmp=}\KeywordTok{acf}\NormalTok{(ar1.sim.}\DecValTok{5}\NormalTok{, }\DataTypeTok{lag.max=}\DecValTok{10}\NormalTok{, }\DataTypeTok{main=}\StringTok{"Sample ACF"}\NormalTok{)}
\end{Highlighting}
\end{Shaded}

\includegraphics{Econ_HW4_files/figure-latex/unnamed-chunk-5-1.pdf}

\begin{Shaded}
\begin{Highlighting}[]
\NormalTok{phi =}\StringTok{ }\FloatTok{0.9}
\NormalTok{mu  =}\StringTok{ }\FloatTok{0.05}
\NormalTok{ar1.model.}\DecValTok{5}\NormalTok{ =}\StringTok{ }\KeywordTok{list}\NormalTok{(}\DataTypeTok{ar=}\NormalTok{phi)}

\KeywordTok{set.seed}\NormalTok{(}\DecValTok{123}\NormalTok{)}
\NormalTok{ar1.sim.}\DecValTok{5}\NormalTok{ =}\StringTok{ }\NormalTok{mu}\OperatorTok{*}\NormalTok{(}\DecValTok{1}\OperatorTok{-}\NormalTok{phi) }\OperatorTok{+}\StringTok{ }\KeywordTok{arima.sim}\NormalTok{(}\DataTypeTok{model=}\NormalTok{ar1.model.}\DecValTok{5}\NormalTok{, }\DataTypeTok{n =} \DecValTok{250}\NormalTok{,}
                             \DataTypeTok{innov=}\KeywordTok{rnorm}\NormalTok{(}\DataTypeTok{n=}\DecValTok{250}\NormalTok{, }\DataTypeTok{mean=}\DecValTok{0}\NormalTok{, }\DataTypeTok{sd=}\FloatTok{0.1}\NormalTok{))}
\NormalTok{acf.ar1.model.}\DecValTok{5}\NormalTok{ =}\StringTok{ }\KeywordTok{ARMAacf}\NormalTok{(}\DataTypeTok{ar=}\NormalTok{phi, }\DataTypeTok{lag.max=}\DecValTok{10}\NormalTok{)}

\KeywordTok{par}\NormalTok{(}\DataTypeTok{mfrow=}\KeywordTok{c}\NormalTok{(}\DecValTok{3}\NormalTok{,}\DecValTok{1}\NormalTok{))}
    \KeywordTok{plot}\NormalTok{(ar1.sim.}\DecValTok{5}\NormalTok{, }\DataTypeTok{main=}\StringTok{"AR(1) Process: mu=0.05, phi=0.9"}\NormalTok{,}
           \DataTypeTok{xlab=}\StringTok{"time"}\NormalTok{,}\DataTypeTok{ylab=}\StringTok{"y(t)"}\NormalTok{)}
    \KeywordTok{abline}\NormalTok{(}\DataTypeTok{h=}\DecValTok{0}\NormalTok{)}
    \KeywordTok{plot}\NormalTok{(}\DecValTok{1}\OperatorTok{:}\DecValTok{11}\NormalTok{, acf.ar1.model.}\DecValTok{5}\NormalTok{, }\DataTypeTok{type=}\StringTok{"h"}\NormalTok{, }\DataTypeTok{col=}\StringTok{"blue"}\NormalTok{, }\DataTypeTok{main=}\StringTok{"Theoretical ACF"}\NormalTok{)}
\NormalTok{    tmp=}\KeywordTok{acf}\NormalTok{(ar1.sim.}\DecValTok{5}\NormalTok{, }\DataTypeTok{lag.max=}\DecValTok{10}\NormalTok{, }\DataTypeTok{main=}\StringTok{"Sample ACF"}\NormalTok{)}
\end{Highlighting}
\end{Shaded}

\includegraphics{Econ_HW4_files/figure-latex/unnamed-chunk-6-1.pdf}

\begin{Shaded}
\begin{Highlighting}[]
\NormalTok{phi =}\StringTok{ }\FloatTok{0.99}
\NormalTok{mu  =}\StringTok{ }\FloatTok{0.05}
\NormalTok{ar1.model.}\DecValTok{5}\NormalTok{ =}\StringTok{ }\KeywordTok{list}\NormalTok{(}\DataTypeTok{ar=}\NormalTok{phi)}

\KeywordTok{set.seed}\NormalTok{(}\DecValTok{123}\NormalTok{)}
\NormalTok{ar1.sim.}\DecValTok{5}\NormalTok{ =}\StringTok{ }\NormalTok{mu}\OperatorTok{*}\NormalTok{(}\DecValTok{1}\OperatorTok{-}\NormalTok{phi) }\OperatorTok{+}\StringTok{ }\KeywordTok{arima.sim}\NormalTok{(}\DataTypeTok{model=}\NormalTok{ar1.model.}\DecValTok{5}\NormalTok{, }\DataTypeTok{n =} \DecValTok{250}\NormalTok{,}
                             \DataTypeTok{innov=}\KeywordTok{rnorm}\NormalTok{(}\DataTypeTok{n=}\DecValTok{250}\NormalTok{, }\DataTypeTok{mean=}\DecValTok{0}\NormalTok{, }\DataTypeTok{sd=}\FloatTok{0.1}\NormalTok{))}
\NormalTok{acf.ar1.model.}\DecValTok{5}\NormalTok{ =}\StringTok{ }\KeywordTok{ARMAacf}\NormalTok{(}\DataTypeTok{ar=}\NormalTok{phi, }\DataTypeTok{lag.max=}\DecValTok{10}\NormalTok{)}

\KeywordTok{par}\NormalTok{(}\DataTypeTok{mfrow=}\KeywordTok{c}\NormalTok{(}\DecValTok{3}\NormalTok{,}\DecValTok{1}\NormalTok{))}
    \KeywordTok{plot}\NormalTok{(ar1.sim.}\DecValTok{5}\NormalTok{, }\DataTypeTok{main=}\StringTok{"AR(1) Process: mu=0.05, phi=0.99"}\NormalTok{,}
           \DataTypeTok{xlab=}\StringTok{"time"}\NormalTok{,}\DataTypeTok{ylab=}\StringTok{"y(t)"}\NormalTok{)}
    \KeywordTok{abline}\NormalTok{(}\DataTypeTok{h=}\DecValTok{0}\NormalTok{)}
    \KeywordTok{plot}\NormalTok{(}\DecValTok{1}\OperatorTok{:}\DecValTok{11}\NormalTok{, acf.ar1.model.}\DecValTok{5}\NormalTok{, }\DataTypeTok{type=}\StringTok{"h"}\NormalTok{, }\DataTypeTok{col=}\StringTok{"blue"}\NormalTok{, }\DataTypeTok{main=}\StringTok{"Theoretical ACF"}\NormalTok{)}
\NormalTok{    tmp=}\KeywordTok{acf}\NormalTok{(ar1.sim.}\DecValTok{5}\NormalTok{, }\DataTypeTok{lag.max=}\DecValTok{10}\NormalTok{, }\DataTypeTok{main=}\StringTok{"Sample ACF"}\NormalTok{)}
\end{Highlighting}
\end{Shaded}

\includegraphics{Econ_HW4_files/figure-latex/unnamed-chunk-7-1.pdf} 2 b)
By observing the ACFs, we ifer that the phi effects the correlation, and
hence dependence in out process. Also, for large values of phi, the
process looks like a non-stationary process. For phi =0, the process
will turn to Y(t) = mu + e(t), thus ACF graph doesn't show any
correlation.

R Exercise B

Question 1

\begin{Shaded}
\begin{Highlighting}[]
\KeywordTok{options}\NormalTok{(}\DataTypeTok{digits=}\DecValTok{4}\NormalTok{, }\DataTypeTok{width=}\DecValTok{70}\NormalTok{)}
\KeywordTok{library}\NormalTok{(PerformanceAnalytics)}
\end{Highlighting}
\end{Shaded}

\begin{verbatim}
## Loading required package: xts
\end{verbatim}

\begin{verbatim}
## Loading required package: zoo
\end{verbatim}

\begin{verbatim}
## 
## Attaching package: 'zoo'
\end{verbatim}

\begin{verbatim}
## The following objects are masked from 'package:base':
## 
##     as.Date, as.Date.numeric
\end{verbatim}

\begin{verbatim}
## 
## Attaching package: 'PerformanceAnalytics'
\end{verbatim}

\begin{verbatim}
## The following object is masked from 'package:graphics':
## 
##     legend
\end{verbatim}

\begin{Shaded}
\begin{Highlighting}[]
\KeywordTok{library}\NormalTok{(zoo)}
\KeywordTok{library}\NormalTok{(tseries)}
\NormalTok{VBLTX.prices =}\StringTok{ }\KeywordTok{get.hist.quote}\NormalTok{(}\DataTypeTok{instrument=}\StringTok{"vbltx"}\NormalTok{, }\DataTypeTok{start=}\StringTok{"1998-01-01"}\NormalTok{,}
                             \DataTypeTok{end=}\StringTok{"2009-12-31"}\NormalTok{, }\DataTypeTok{quote=}\StringTok{"AdjClose"}\NormalTok{,}
                             \DataTypeTok{provider=}\StringTok{"yahoo"}\NormalTok{, }\DataTypeTok{origin=}\StringTok{"1970-01-01"}\NormalTok{,}
                             \DataTypeTok{compression=}\StringTok{"m"}\NormalTok{, }\DataTypeTok{retclass=}\StringTok{"zoo"}\NormalTok{)}
\end{Highlighting}
\end{Shaded}

\begin{verbatim}
## 'getSymbols' currently uses auto.assign=TRUE by default, but will
## use auto.assign=FALSE in 0.5-0. You will still be able to use
## 'loadSymbols' to automatically load data. getOption("getSymbols.env")
## and getOption("getSymbols.auto.assign") will still be checked for
## alternate defaults.
## 
## This message is shown once per session and may be disabled by setting 
## options("getSymbols.warning4.0"=FALSE). See ?getSymbols for details.
\end{verbatim}

\begin{verbatim}
## 
## WARNING: There have been significant changes to Yahoo Finance data.
## Please see the Warning section of '?getSymbols.yahoo' for details.
## 
## This message is shown once per session and may be disabled by setting
## options("getSymbols.yahoo.warning"=FALSE).
\end{verbatim}

\begin{verbatim}
## time series ends   2009-12-01
\end{verbatim}

\begin{Shaded}
\begin{Highlighting}[]
\KeywordTok{index}\NormalTok{(VBLTX.prices) =}\StringTok{ }\KeywordTok{as.yearmon}\NormalTok{(}\KeywordTok{index}\NormalTok{(VBLTX.prices))}

\NormalTok{FMAGX.prices =}\StringTok{ }\KeywordTok{get.hist.quote}\NormalTok{(}\DataTypeTok{instrument=}\StringTok{"fmagx"}\NormalTok{, }\DataTypeTok{start=}\StringTok{"1998-01-01"}\NormalTok{,}
                             \DataTypeTok{end=}\StringTok{"2009-12-31"}\NormalTok{, }\DataTypeTok{quote=}\StringTok{"AdjClose"}\NormalTok{,}
                             \DataTypeTok{provider=}\StringTok{"yahoo"}\NormalTok{, }\DataTypeTok{origin=}\StringTok{"1970-01-01"}\NormalTok{,}
                             \DataTypeTok{compression=}\StringTok{"m"}\NormalTok{, }\DataTypeTok{retclass=}\StringTok{"zoo"}\NormalTok{)}
\end{Highlighting}
\end{Shaded}

\begin{verbatim}
## time series ends   2009-12-01
\end{verbatim}

\begin{Shaded}
\begin{Highlighting}[]
\KeywordTok{index}\NormalTok{(FMAGX.prices) =}\StringTok{ }\KeywordTok{as.yearmon}\NormalTok{(}\KeywordTok{index}\NormalTok{(FMAGX.prices))}

\NormalTok{SBUX.prices =}\StringTok{ }\KeywordTok{get.hist.quote}\NormalTok{(}\DataTypeTok{instrument=}\StringTok{"sbux"}\NormalTok{, }\DataTypeTok{start=}\StringTok{"1998-01-01"}\NormalTok{,}
                             \DataTypeTok{end=}\StringTok{"2009-12-31"}\NormalTok{, }\DataTypeTok{quote=}\StringTok{"AdjClose"}\NormalTok{,}
                             \DataTypeTok{provider=}\StringTok{"yahoo"}\NormalTok{, }\DataTypeTok{origin=}\StringTok{"1970-01-01"}\NormalTok{,}
                             \DataTypeTok{compression=}\StringTok{"m"}\NormalTok{, }\DataTypeTok{retclass=}\StringTok{"zoo"}\NormalTok{)}
\end{Highlighting}
\end{Shaded}

\begin{verbatim}
## time series ends   2009-12-01
\end{verbatim}

\begin{Shaded}
\begin{Highlighting}[]
\KeywordTok{index}\NormalTok{(SBUX.prices) =}\StringTok{ }\KeywordTok{as.yearmon}\NormalTok{(}\KeywordTok{index}\NormalTok{(SBUX.prices))}
\CommentTok{#merged data}
\NormalTok{lab4Prices.z =}\StringTok{ }\KeywordTok{merge}\NormalTok{(VBLTX.prices, FMAGX.prices, SBUX.prices)}
\KeywordTok{colnames}\NormalTok{(lab4Prices.z) =}\StringTok{ }\KeywordTok{c}\NormalTok{(}\StringTok{"VBLTX"}\NormalTok{, }\StringTok{"FMAGX"}\NormalTok{, }\StringTok{"SBUX"}\NormalTok{)}
\NormalTok{lab4Returns.z =}\StringTok{ }\KeywordTok{diff}\NormalTok{(}\KeywordTok{log}\NormalTok{(lab4Prices.z))}

\KeywordTok{start}\NormalTok{(lab4Returns.z)}
\end{Highlighting}
\end{Shaded}

\begin{verbatim}
## [1] "Feb 1998"
\end{verbatim}

\begin{Shaded}
\begin{Highlighting}[]
\KeywordTok{end}\NormalTok{(lab4Returns.z)}
\end{Highlighting}
\end{Shaded}

\begin{verbatim}
## [1] "Dec 2009"
\end{verbatim}

\begin{Shaded}
\begin{Highlighting}[]
\KeywordTok{colnames}\NormalTok{(lab4Returns.z) }
\end{Highlighting}
\end{Shaded}

\begin{verbatim}
## [1] "VBLTX" "FMAGX" "SBUX"
\end{verbatim}

\begin{Shaded}
\begin{Highlighting}[]
\KeywordTok{head}\NormalTok{(lab4Returns.z)}
\end{Highlighting}
\end{Shaded}

\begin{verbatim}
##              VBLTX     FMAGX      SBUX
## Feb 1998 -0.004881  0.073069  0.078858
## Mar 1998  0.001037  0.049066  0.135702
## Apr 1998  0.006181  0.011513  0.060219
## May 1998  0.018136 -0.045728 -0.002601
## Jun 1998  0.020749  0.067677  0.107312
## Jul 1998 -0.005911 -0.007508 -0.243824
\end{verbatim}

1 a)

\begin{Shaded}
\begin{Highlighting}[]
\NormalTok{my.panel <-}\StringTok{ }\ControlFlowTok{function}\NormalTok{(...) \{}
  \KeywordTok{lines}\NormalTok{(...)}
  \KeywordTok{abline}\NormalTok{(}\DataTypeTok{h=}\DecValTok{0}\NormalTok{)}
\NormalTok{\}}
\KeywordTok{plot}\NormalTok{(lab4Returns.z, }\DataTypeTok{col=}\StringTok{"blue"}\NormalTok{, }\DataTypeTok{lwd=}\DecValTok{2}\NormalTok{, }\DataTypeTok{main=}\StringTok{"Monthly cc returns on 3 assets"}\NormalTok{,}
     \DataTypeTok{panel=}\NormalTok{my.panel)}
\end{Highlighting}
\end{Shaded}

\includegraphics{Econ_HW4_files/figure-latex/unnamed-chunk-9-1.pdf}

The scale for each graph is different. The graph of VBLTX has lowest
volatility, and SBUX has highest volatility if we conisder the sclae
difference. In 2018, VBlTX went up, but other two funds went down.

\begin{Shaded}
\begin{Highlighting}[]
\KeywordTok{chart.TimeSeries}\NormalTok{(lab4Returns.z, }\DataTypeTok{legend.loc=}\StringTok{"bottom"}\NormalTok{, }\DataTypeTok{main=}\StringTok{""}\NormalTok{) }
\end{Highlighting}
\end{Shaded}

\includegraphics{Econ_HW4_files/figure-latex/unnamed-chunk-10-1.pdf} We
can observe the volatility better by comparing them all on the same
graph.

\begin{Shaded}
\begin{Highlighting}[]
\KeywordTok{chart.Bar}\NormalTok{(lab4Returns.z, }\DataTypeTok{main=}\StringTok{""}\NormalTok{)}
\end{Highlighting}
\end{Shaded}

\begin{verbatim}
## Warning in plot.xts(x = y, y = NULL, ..., col = colorset, type = type,
## lty = lty, : only the univariate series will be plotted
\end{verbatim}

\includegraphics{Econ_HW4_files/figure-latex/unnamed-chunk-11-1.pdf}

1 b)

\begin{Shaded}
\begin{Highlighting}[]
\KeywordTok{chart.CumReturns}\NormalTok{(}\KeywordTok{diff}\NormalTok{(lab4Prices.z)}\OperatorTok{/}\KeywordTok{lag}\NormalTok{(lab4Prices.z, }\DataTypeTok{k=}\OperatorTok{-}\DecValTok{1}\NormalTok{), }
                 \DataTypeTok{legend.loc=}\StringTok{"topleft"}\NormalTok{, }\DataTypeTok{wealth.index=}\OtherTok{TRUE}\NormalTok{,}
                 \DataTypeTok{main=}\StringTok{"Future Value of $1 invested"}\NormalTok{)}
\end{Highlighting}
\end{Shaded}

\includegraphics{Econ_HW4_files/figure-latex/unnamed-chunk-12-1.pdf}

\begin{Shaded}
\begin{Highlighting}[]
\NormalTok{ret.mat =}\StringTok{ }\KeywordTok{coredata}\NormalTok{(lab4Returns.z)}
\end{Highlighting}
\end{Shaded}

Over the investment horizon, SBUX had the best future value, but also
had highest volatility VBLTX had steady return. FMAGX had the worst
future value.

1 c)

\begin{Shaded}
\begin{Highlighting}[]
\KeywordTok{par}\NormalTok{(}\DataTypeTok{mfrow=}\KeywordTok{c}\NormalTok{(}\DecValTok{2}\NormalTok{,}\DecValTok{2}\NormalTok{))}
    \KeywordTok{hist}\NormalTok{(ret.mat[,}\StringTok{"VBLTX"}\NormalTok{],}\DataTypeTok{main=}\StringTok{"VBLTX monthly returns"}\NormalTok{,}
         \DataTypeTok{xlab=}\StringTok{"VBLTX"}\NormalTok{, }\DataTypeTok{probability=}\NormalTok{T, }\DataTypeTok{col=}\StringTok{"slateblue1"}\NormalTok{)}
    \KeywordTok{boxplot}\NormalTok{(ret.mat[,}\StringTok{"VBLTX"}\NormalTok{],}\DataTypeTok{outchar=}\NormalTok{T, }\DataTypeTok{main=}\StringTok{"Boxplot"}\NormalTok{, }\DataTypeTok{col=}\StringTok{"slateblue1"}\NormalTok{)}
    \KeywordTok{plot}\NormalTok{(}\KeywordTok{density}\NormalTok{(ret.mat[,}\StringTok{"VBLTX"}\NormalTok{]),}\DataTypeTok{type=}\StringTok{"l"}\NormalTok{, }\DataTypeTok{main=}\StringTok{"Smoothed density"}\NormalTok{,}
           \DataTypeTok{xlab=}\StringTok{"monthly return"}\NormalTok{, }\DataTypeTok{ylab=}\StringTok{"density estimate"}\NormalTok{, }\DataTypeTok{col=}\StringTok{"slateblue1"}\NormalTok{)}
    \KeywordTok{qqnorm}\NormalTok{(ret.mat[,}\StringTok{"VBLTX"}\NormalTok{], }\DataTypeTok{col=}\StringTok{"slateblue1"}\NormalTok{)}
    \KeywordTok{qqline}\NormalTok{(ret.mat[,}\StringTok{"VBLTX"}\NormalTok{])}
\end{Highlighting}
\end{Shaded}

\includegraphics{Econ_HW4_files/figure-latex/unnamed-chunk-13-1.pdf}

\begin{Shaded}
\begin{Highlighting}[]
\KeywordTok{par}\NormalTok{(}\DataTypeTok{mfrow=}\KeywordTok{c}\NormalTok{(}\DecValTok{1}\NormalTok{,}\DecValTok{1}\NormalTok{))}
\end{Highlighting}
\end{Shaded}

The distribution is relatively symmetric with fatter tails on the left
side as we see it on box-plot and hist density. The Q-Q plot indicates
some differences on the tails, but pretty much like normal on the
middle.

\begin{Shaded}
\begin{Highlighting}[]
\KeywordTok{par}\NormalTok{(}\DataTypeTok{mfrow=}\KeywordTok{c}\NormalTok{(}\DecValTok{2}\NormalTok{,}\DecValTok{2}\NormalTok{))}
    \KeywordTok{hist}\NormalTok{(ret.mat[,}\StringTok{"FMAGX"}\NormalTok{],}\DataTypeTok{main=}\StringTok{"FMAGX monthly returns"}\NormalTok{,}
         \DataTypeTok{xlab=}\StringTok{"FMAGX"}\NormalTok{, }\DataTypeTok{probability=}\NormalTok{T, }\DataTypeTok{col=}\StringTok{"slateblue1"}\NormalTok{)}
    \KeywordTok{boxplot}\NormalTok{(ret.mat[,}\StringTok{"FMAGX"}\NormalTok{],}\DataTypeTok{outchar=}\NormalTok{T, }\DataTypeTok{main=}\StringTok{"Boxplot"}\NormalTok{, }\DataTypeTok{col=}\StringTok{"slateblue1"}\NormalTok{)}
    \KeywordTok{plot}\NormalTok{(}\KeywordTok{density}\NormalTok{(ret.mat[,}\StringTok{"FMAGX"}\NormalTok{]),}\DataTypeTok{type=}\StringTok{"l"}\NormalTok{, }\DataTypeTok{main=}\StringTok{"Smoothed density"}\NormalTok{,}
         \DataTypeTok{xlab=}\StringTok{"monthly return"}\NormalTok{, }\DataTypeTok{ylab=}\StringTok{"density estimate"}\NormalTok{, }\DataTypeTok{col=}\StringTok{"slateblue1"}\NormalTok{)}
    \KeywordTok{qqnorm}\NormalTok{(ret.mat[,}\StringTok{"FMAGX"}\NormalTok{], }\DataTypeTok{col=}\StringTok{"slateblue1"}\NormalTok{)}
    \KeywordTok{qqline}\NormalTok{(ret.mat[,}\StringTok{"FMAGX"}\NormalTok{])}
\end{Highlighting}
\end{Shaded}

\includegraphics{Econ_HW4_files/figure-latex/unnamed-chunk-14-1.pdf}

\begin{Shaded}
\begin{Highlighting}[]
\KeywordTok{par}\NormalTok{(}\DataTypeTok{mfrow=}\KeywordTok{c}\NormalTok{(}\DecValTok{1}\NormalTok{,}\DecValTok{1}\NormalTok{))}
\end{Highlighting}
\end{Shaded}

The distribution doesn't look normal.(more data on the right, negatively
skewed) The Q-Q plot indicates fat tail on the left side than normal,
but simillar to normal on the middle.

\begin{Shaded}
\begin{Highlighting}[]
\KeywordTok{par}\NormalTok{(}\DataTypeTok{mfrow=}\KeywordTok{c}\NormalTok{(}\DecValTok{2}\NormalTok{,}\DecValTok{2}\NormalTok{))}
    \KeywordTok{hist}\NormalTok{(ret.mat[,}\StringTok{"SBUX"}\NormalTok{],}\DataTypeTok{main=}\StringTok{"SBUX monthly returns"}\NormalTok{,}
         \DataTypeTok{xlab=}\StringTok{"SBUX"}\NormalTok{, }\DataTypeTok{probability=}\NormalTok{T, }\DataTypeTok{col=}\StringTok{"slateblue1"}\NormalTok{)}
    \KeywordTok{boxplot}\NormalTok{(ret.mat[,}\StringTok{"SBUX"}\NormalTok{],}\DataTypeTok{outchar=}\NormalTok{T, }\DataTypeTok{main=}\StringTok{"Boxplot"}\NormalTok{, }\DataTypeTok{col=}\StringTok{"slateblue1"}\NormalTok{)}
    \KeywordTok{plot}\NormalTok{(}\KeywordTok{density}\NormalTok{(ret.mat[,}\StringTok{"SBUX"}\NormalTok{]),}\DataTypeTok{type=}\StringTok{"l"}\NormalTok{, }\DataTypeTok{main=}\StringTok{"Smoothed density"}\NormalTok{,}
         \DataTypeTok{xlab=}\StringTok{"monthly return"}\NormalTok{, }\DataTypeTok{ylab=}\StringTok{"density estimate"}\NormalTok{, }\DataTypeTok{col=}\StringTok{"slateblue1"}\NormalTok{)}
    \KeywordTok{qqnorm}\NormalTok{(ret.mat[,}\StringTok{"SBUX"}\NormalTok{], }\DataTypeTok{col=}\StringTok{"slateblue1"}\NormalTok{)}
    \KeywordTok{qqline}\NormalTok{(ret.mat[,}\StringTok{"SBUX"}\NormalTok{])}
\end{Highlighting}
\end{Shaded}

\includegraphics{Econ_HW4_files/figure-latex/unnamed-chunk-15-1.pdf}

\begin{Shaded}
\begin{Highlighting}[]
\KeywordTok{par}\NormalTok{(}\DataTypeTok{mfrow=}\KeywordTok{c}\NormalTok{(}\DecValTok{1}\NormalTok{,}\DecValTok{1}\NormalTok{))}
\end{Highlighting}
\end{Shaded}

The distribution is not very symmetric with(somehow negatively skewed)
The Q-Q plot indicated left tail doesn't look like normal but right is
more simillar

\begin{Shaded}
\begin{Highlighting}[]
\KeywordTok{boxplot}\NormalTok{(ret.mat[,}\StringTok{"VBLTX"}\NormalTok{], ret.mat[,}\StringTok{"FMAGX"}\NormalTok{], ret.mat[,}\StringTok{"SBUX"}\NormalTok{],}
        \DataTypeTok{names=}\KeywordTok{colnames}\NormalTok{(ret.mat), }\DataTypeTok{col=}\StringTok{"slateblue1"}\NormalTok{)}
\end{Highlighting}
\end{Shaded}

\includegraphics{Econ_HW4_files/figure-latex/unnamed-chunk-16-1.pdf}
This box-plot graph will help us to compare them all together b/c they
are on the same scale. As we can see, VBLTX has the lowest volatility,
and SBUX has the highest.

1 d)

\begin{Shaded}
\begin{Highlighting}[]
\KeywordTok{summary}\NormalTok{(ret.mat)}
\end{Highlighting}
\end{Shaded}

\begin{verbatim}
##      VBLTX              FMAGX               SBUX        
##  Min.   :-0.09138   Min.   :-0.26430   Min.   :-0.4797  
##  1st Qu.:-0.00958   1st Qu.:-0.02124   1st Qu.:-0.0488  
##  Median : 0.00855   Median : 0.00825   Median : 0.0182  
##  Mean   : 0.00530   Mean   : 0.00186   Mean   : 0.0113  
##  3rd Qu.: 0.02038   3rd Qu.: 0.04503   3rd Qu.: 0.0868  
##  Max.   : 0.10827   Max.   : 0.20686   Max.   : 0.2773
\end{verbatim}

\begin{Shaded}
\begin{Highlighting}[]
\KeywordTok{print}\NormalTok{(}\StringTok{"Mean"}\NormalTok{)}
\end{Highlighting}
\end{Shaded}

\begin{verbatim}
## [1] "Mean"
\end{verbatim}

\begin{Shaded}
\begin{Highlighting}[]
\KeywordTok{apply}\NormalTok{(ret.mat, }\DecValTok{2}\NormalTok{, mean)}
\end{Highlighting}
\end{Shaded}

\begin{verbatim}
##    VBLTX    FMAGX     SBUX 
## 0.005302 0.001856 0.011318
\end{verbatim}

\begin{Shaded}
\begin{Highlighting}[]
\KeywordTok{print}\NormalTok{(}\StringTok{"var"}\NormalTok{)}
\end{Highlighting}
\end{Shaded}

\begin{verbatim}
## [1] "var"
\end{verbatim}

\begin{Shaded}
\begin{Highlighting}[]
\KeywordTok{apply}\NormalTok{(ret.mat, }\DecValTok{2}\NormalTok{, var)}
\end{Highlighting}
\end{Shaded}

\begin{verbatim}
##     VBLTX     FMAGX      SBUX 
## 0.0006903 0.0041077 0.0142545
\end{verbatim}

\begin{Shaded}
\begin{Highlighting}[]
\KeywordTok{print}\NormalTok{(}\StringTok{"SD"}\NormalTok{)}
\end{Highlighting}
\end{Shaded}

\begin{verbatim}
## [1] "SD"
\end{verbatim}

\begin{Shaded}
\begin{Highlighting}[]
\KeywordTok{apply}\NormalTok{(ret.mat, }\DecValTok{2}\NormalTok{, sd)}
\end{Highlighting}
\end{Shaded}

\begin{verbatim}
##   VBLTX   FMAGX    SBUX 
## 0.02627 0.06409 0.11939
\end{verbatim}

\begin{Shaded}
\begin{Highlighting}[]
\KeywordTok{print}\NormalTok{(}\StringTok{"skewness"}\NormalTok{)}
\end{Highlighting}
\end{Shaded}

\begin{verbatim}
## [1] "skewness"
\end{verbatim}

\begin{Shaded}
\begin{Highlighting}[]
\KeywordTok{apply}\NormalTok{(ret.mat, }\DecValTok{2}\NormalTok{, skewness)}
\end{Highlighting}
\end{Shaded}

\begin{verbatim}
##   VBLTX   FMAGX    SBUX 
## -0.1470 -0.9284 -0.9061
\end{verbatim}

\begin{Shaded}
\begin{Highlighting}[]
\KeywordTok{print}\NormalTok{(}\StringTok{"kurtosis"}\NormalTok{)}
\end{Highlighting}
\end{Shaded}

\begin{verbatim}
## [1] "kurtosis"
\end{verbatim}

\begin{Shaded}
\begin{Highlighting}[]
\KeywordTok{apply}\NormalTok{(ret.mat, }\DecValTok{2}\NormalTok{, kurtosis)}
\end{Highlighting}
\end{Shaded}

\begin{verbatim}
## VBLTX FMAGX  SBUX 
## 2.887 3.398 2.695
\end{verbatim}

SBUX has the highest mean, and all the means are positive. Variance of
SUBUX is the highest, and VBLTX has the lowest variance which we
observed it in the previous parts. Skeweness is negative for all of
them. Meaning that distributions are all negatively skewed. Kurtosis
(excess) indicates that they all have fatter tails than nomral.

1 e) and f)

\begin{Shaded}
\begin{Highlighting}[]
\CommentTok{# annualized cc mean }
\DecValTok{12}\OperatorTok{*}\KeywordTok{apply}\NormalTok{(ret.mat, }\DecValTok{2}\NormalTok{, mean)}
\end{Highlighting}
\end{Shaded}

\begin{verbatim}
##   VBLTX   FMAGX    SBUX 
## 0.06363 0.02227 0.13582
\end{verbatim}

\begin{Shaded}
\begin{Highlighting}[]
\CommentTok{# annualized simple mean}
\KeywordTok{exp}\NormalTok{(}\DecValTok{12}\OperatorTok{*}\KeywordTok{apply}\NormalTok{(ret.mat, }\DecValTok{2}\NormalTok{, mean)) }\OperatorTok{-}\StringTok{ }\DecValTok{1}
\end{Highlighting}
\end{Shaded}

\begin{verbatim}
##   VBLTX   FMAGX    SBUX 
## 0.06569 0.02252 0.14548
\end{verbatim}

\begin{Shaded}
\begin{Highlighting}[]
\CommentTok{# annualized sd values}
\KeywordTok{sqrt}\NormalTok{(}\DecValTok{12}\NormalTok{)}\OperatorTok{*}\KeywordTok{apply}\NormalTok{(ret.mat, }\DecValTok{2}\NormalTok{, sd)}
\end{Highlighting}
\end{Shaded}

\begin{verbatim}
##   VBLTX   FMAGX    SBUX 
## 0.09101 0.22202 0.41359
\end{verbatim}

first row is the cc return. SBUX has the highest(but with high sd).
VBLTX has the lowest mean. The data makes sense because stock the most
volatile(more than portfolios), and portfolios are also more volatile
than bonds. sd value for SBUX is the highest, and lowest for VBlTX.

1 g)

\begin{Shaded}
\begin{Highlighting}[]
\KeywordTok{pairs}\NormalTok{(ret.mat, }\DataTypeTok{col=}\StringTok{"slateblue1"}\NormalTok{, }\DataTypeTok{pch=}\DecValTok{16}\NormalTok{)}
\end{Highlighting}
\end{Shaded}

\includegraphics{Econ_HW4_files/figure-latex/unnamed-chunk-19-1.pdf} No
clear relation exists except between FMAGX and SBUX which a linear
relationship can be seen.

1 h)

\begin{Shaded}
\begin{Highlighting}[]
\CommentTok{# compute 3 x 3 covariance and correlation matrices}
\KeywordTok{var}\NormalTok{(ret.mat)}
\end{Highlighting}
\end{Shaded}

\begin{verbatim}
##            VBLTX     FMAGX       SBUX
## VBLTX  0.0006903 0.0001074 -0.0001761
## FMAGX  0.0001074 0.0041077  0.0032432
## SBUX  -0.0001761 0.0032432  0.0142545
\end{verbatim}

\begin{Shaded}
\begin{Highlighting}[]
\KeywordTok{cor}\NormalTok{(ret.mat)}
\end{Highlighting}
\end{Shaded}

\begin{verbatim}
##          VBLTX   FMAGX     SBUX
## VBLTX  1.00000 0.06376 -0.05614
## FMAGX  0.06376 1.00000  0.42384
## SBUX  -0.05614 0.42384  1.00000
\end{verbatim}

cov between VBLTX and FMAGX is positive (positive linear relation) cov
between FMAGX and SBUX is positive (positive linear relation) cov
between VBLTX and SBUX is negative (negative linear relation)

\begin{Shaded}
\begin{Highlighting}[]
\KeywordTok{par}\NormalTok{(}\DataTypeTok{mfrow=}\KeywordTok{c}\NormalTok{(}\DecValTok{3}\NormalTok{,}\DecValTok{1}\NormalTok{))}
\NormalTok{    acf.msft =}\StringTok{ }\KeywordTok{acf}\NormalTok{(ret.mat[,}\StringTok{"VBLTX"}\NormalTok{], }\DataTypeTok{main=}\StringTok{"VBLTX"}\NormalTok{)}
\NormalTok{    acf.sbux =}\StringTok{ }\KeywordTok{acf}\NormalTok{(ret.mat[,}\StringTok{"FMAGX"}\NormalTok{], }\DataTypeTok{main=}\StringTok{"FMAGX"}\NormalTok{)}
\NormalTok{    acf.sp500 =}\StringTok{ }\KeywordTok{acf}\NormalTok{(ret.mat[,}\StringTok{"SBUX"}\NormalTok{], }\DataTypeTok{main=}\StringTok{"SBUX"}\NormalTok{)}
\end{Highlighting}
\end{Shaded}

\includegraphics{Econ_HW4_files/figure-latex/unnamed-chunk-21-1.pdf}

\begin{Shaded}
\begin{Highlighting}[]
\KeywordTok{par}\NormalTok{(}\DataTypeTok{mfrow=}\KeywordTok{c}\NormalTok{(}\DecValTok{1}\NormalTok{,}\DecValTok{1}\NormalTok{))}
\end{Highlighting}
\end{Shaded}

It shows that it is almost uncorrelated for all 3 assets.

2 a)

\begin{Shaded}
\begin{Highlighting}[]
\NormalTok{muhat.vals =}\StringTok{ }\KeywordTok{apply}\NormalTok{(ret.mat, }\DecValTok{2}\NormalTok{, mean)}
\NormalTok{muhat.vals}
\end{Highlighting}
\end{Shaded}

\begin{verbatim}
##    VBLTX    FMAGX     SBUX 
## 0.005302 0.001856 0.011318
\end{verbatim}

\begin{Shaded}
\begin{Highlighting}[]
\NormalTok{sigma2hat.vals =}\StringTok{ }\KeywordTok{apply}\NormalTok{(ret.mat, }\DecValTok{2}\NormalTok{, var)}
\NormalTok{sigma2hat.vals}
\end{Highlighting}
\end{Shaded}

\begin{verbatim}
##     VBLTX     FMAGX      SBUX 
## 0.0006903 0.0041077 0.0142545
\end{verbatim}

\begin{Shaded}
\begin{Highlighting}[]
\NormalTok{sigmahat.vals =}\StringTok{ }\KeywordTok{apply}\NormalTok{(ret.mat, }\DecValTok{2}\NormalTok{, sd)}
\NormalTok{sigmahat.vals}
\end{Highlighting}
\end{Shaded}

\begin{verbatim}
##   VBLTX   FMAGX    SBUX 
## 0.02627 0.06409 0.11939
\end{verbatim}

\begin{Shaded}
\begin{Highlighting}[]
\NormalTok{cov.mat =}\StringTok{ }\KeywordTok{var}\NormalTok{(ret.mat)}
\NormalTok{cov.mat}
\end{Highlighting}
\end{Shaded}

\begin{verbatim}
##            VBLTX     FMAGX       SBUX
## VBLTX  0.0006903 0.0001074 -0.0001761
## FMAGX  0.0001074 0.0041077  0.0032432
## SBUX  -0.0001761 0.0032432  0.0142545
\end{verbatim}

\begin{Shaded}
\begin{Highlighting}[]
\NormalTok{cor.mat =}\StringTok{ }\KeywordTok{cor}\NormalTok{(ret.mat)}
\NormalTok{cor.mat}
\end{Highlighting}
\end{Shaded}

\begin{verbatim}
##          VBLTX   FMAGX     SBUX
## VBLTX  1.00000 0.06376 -0.05614
## FMAGX  0.06376 1.00000  0.42384
## SBUX  -0.05614 0.42384  1.00000
\end{verbatim}

\begin{Shaded}
\begin{Highlighting}[]
\NormalTok{covhat.vals =}\StringTok{ }\NormalTok{cov.mat[}\KeywordTok{lower.tri}\NormalTok{(cov.mat)]}
\NormalTok{rhohat.vals =}\StringTok{ }\NormalTok{cor.mat[}\KeywordTok{lower.tri}\NormalTok{(cor.mat)]}
\KeywordTok{names}\NormalTok{(covhat.vals) <-}\StringTok{ }\KeywordTok{names}\NormalTok{(rhohat.vals) <-}\StringTok{ }
\KeywordTok{c}\NormalTok{(}\StringTok{"VBLTX,FMAGX"}\NormalTok{,}\StringTok{"VBLTX,SBUX"}\NormalTok{,}\StringTok{"FMAGX,SBUX"}\NormalTok{)}
\NormalTok{covhat.vals}
\end{Highlighting}
\end{Shaded}

\begin{verbatim}
## VBLTX,FMAGX  VBLTX,SBUX  FMAGX,SBUX 
##   0.0001074  -0.0001761   0.0032432
\end{verbatim}

\begin{Shaded}
\begin{Highlighting}[]
\NormalTok{rhohat.vals}
\end{Highlighting}
\end{Shaded}

\begin{verbatim}
## VBLTX,FMAGX  VBLTX,SBUX  FMAGX,SBUX 
##     0.06376    -0.05614     0.42384
\end{verbatim}

\begin{Shaded}
\begin{Highlighting}[]
\KeywordTok{cbind}\NormalTok{(muhat.vals,sigma2hat.vals,sigmahat.vals)}
\end{Highlighting}
\end{Shaded}

\begin{verbatim}
##       muhat.vals sigma2hat.vals sigmahat.vals
## VBLTX   0.005302      0.0006903       0.02627
## FMAGX   0.001856      0.0041077       0.06409
## SBUX    0.011318      0.0142545       0.11939
\end{verbatim}

\begin{Shaded}
\begin{Highlighting}[]
\KeywordTok{cbind}\NormalTok{(covhat.vals,rhohat.vals)}
\end{Highlighting}
\end{Shaded}

\begin{verbatim}
##             covhat.vals rhohat.vals
## VBLTX,FMAGX   0.0001074     0.06376
## VBLTX,SBUX   -0.0001761    -0.05614
## FMAGX,SBUX    0.0032432     0.42384
\end{verbatim}

As we can see, SBUX has the highest mean and SD. (make sense b/c it's
individual stock) VBLTX has the lowest SD b/c it's a bond. FAMAGX is a
portfolio which mean the lowest.

2 b) c)

\begin{Shaded}
\begin{Highlighting}[]
\CommentTok{# compute estimated standard error for mean}
\NormalTok{nobs =}\StringTok{ }\KeywordTok{nrow}\NormalTok{(ret.mat)}
\NormalTok{nobs}
\end{Highlighting}
\end{Shaded}

\begin{verbatim}
## [1] 143
\end{verbatim}

\begin{Shaded}
\begin{Highlighting}[]
\NormalTok{se.muhat =}\StringTok{ }\NormalTok{sigmahat.vals}\OperatorTok{/}\KeywordTok{sqrt}\NormalTok{(nobs)}
\NormalTok{se.muhat}
\end{Highlighting}
\end{Shaded}

\begin{verbatim}
##    VBLTX    FMAGX     SBUX 
## 0.002197 0.005360 0.009984
\end{verbatim}

\begin{Shaded}
\begin{Highlighting}[]
\KeywordTok{cbind}\NormalTok{(muhat.vals,se.muhat)}
\end{Highlighting}
\end{Shaded}

\begin{verbatim}
##       muhat.vals se.muhat
## VBLTX   0.005302 0.002197
## FMAGX   0.001856 0.005360
## SBUX    0.011318 0.009984
\end{verbatim}

\begin{Shaded}
\begin{Highlighting}[]
\CommentTok{# compute approx 95% confidence intervals}
\NormalTok{mu.lower =}\StringTok{ }\NormalTok{muhat.vals }\OperatorTok{-}\StringTok{ }\DecValTok{2}\OperatorTok{*}\NormalTok{se.muhat}
\NormalTok{mu.upper =}\StringTok{ }\NormalTok{muhat.vals }\OperatorTok{+}\StringTok{ }\DecValTok{2}\OperatorTok{*}\NormalTok{se.muhat}
\KeywordTok{cbind}\NormalTok{(mu.lower,mu.upper)}
\end{Highlighting}
\end{Shaded}

\begin{verbatim}
##        mu.lower mu.upper
## VBLTX  0.000908 0.009696
## FMAGX -0.008863 0.012575
## SBUX  -0.008650 0.031286
\end{verbatim}

\begin{Shaded}
\begin{Highlighting}[]
\CommentTok{# compute estimated standard errors for variance and sd}
\NormalTok{se.sigma2hat =}\StringTok{ }\NormalTok{sigma2hat.vals}\OperatorTok{/}\KeywordTok{sqrt}\NormalTok{(nobs}\OperatorTok{/}\DecValTok{2}\NormalTok{)}
\NormalTok{se.sigma2hat}
\end{Highlighting}
\end{Shaded}

\begin{verbatim}
##     VBLTX     FMAGX      SBUX 
## 8.163e-05 4.858e-04 1.686e-03
\end{verbatim}

\begin{Shaded}
\begin{Highlighting}[]
\NormalTok{se.sigmahat =}\StringTok{ }\NormalTok{sigmahat.vals}\OperatorTok{/}\KeywordTok{sqrt}\NormalTok{(}\DecValTok{2}\OperatorTok{*}\NormalTok{nobs)}
\NormalTok{se.sigmahat}
\end{Highlighting}
\end{Shaded}

\begin{verbatim}
##    VBLTX    FMAGX     SBUX 
## 0.001554 0.003790 0.007060
\end{verbatim}

\begin{Shaded}
\begin{Highlighting}[]
\KeywordTok{cbind}\NormalTok{(sigma2hat.vals,se.sigma2hat)}
\end{Highlighting}
\end{Shaded}

\begin{verbatim}
##       sigma2hat.vals se.sigma2hat
## VBLTX      0.0006903    8.163e-05
## FMAGX      0.0041077    4.858e-04
## SBUX       0.0142545    1.686e-03
\end{verbatim}

\begin{Shaded}
\begin{Highlighting}[]
\KeywordTok{cbind}\NormalTok{(sigmahat.vals,se.sigmahat)}
\end{Highlighting}
\end{Shaded}

\begin{verbatim}
##       sigmahat.vals se.sigmahat
## VBLTX       0.02627    0.001554
## FMAGX       0.06409    0.003790
## SBUX        0.11939    0.007060
\end{verbatim}

\begin{Shaded}
\begin{Highlighting}[]
\CommentTok{# compute approx 95% confidence intervals}
\NormalTok{sigma2.lower =}\StringTok{ }\NormalTok{sigma2hat.vals }\OperatorTok{-}\StringTok{ }\DecValTok{2}\OperatorTok{*}\NormalTok{se.sigma2hat}
\NormalTok{sigma2.upper =}\StringTok{ }\NormalTok{sigma2hat.vals }\OperatorTok{+}\StringTok{ }\DecValTok{2}\OperatorTok{*}\NormalTok{se.sigma2hat}
\KeywordTok{cbind}\NormalTok{(sigma2.lower,sigma2.upper)}
\end{Highlighting}
\end{Shaded}

\begin{verbatim}
##       sigma2.lower sigma2.upper
## VBLTX     0.000527    0.0008535
## FMAGX     0.003136    0.0050793
## SBUX      0.010883    0.0176260
\end{verbatim}

\begin{Shaded}
\begin{Highlighting}[]
\NormalTok{sigma.lower =}\StringTok{ }\NormalTok{sigmahat.vals }\OperatorTok{-}\StringTok{ }\DecValTok{2}\OperatorTok{*}\NormalTok{se.sigmahat}
\NormalTok{sigma.upper =}\StringTok{ }\NormalTok{sigmahat.vals }\OperatorTok{+}\StringTok{ }\DecValTok{2}\OperatorTok{*}\NormalTok{se.sigmahat}
\KeywordTok{cbind}\NormalTok{(sigma.lower,sigma.upper)}
\end{Highlighting}
\end{Shaded}

\begin{verbatim}
##       sigma.lower sigma.upper
## VBLTX     0.02317     0.02938
## FMAGX     0.05651     0.07167
## SBUX      0.10527     0.13351
\end{verbatim}

\begin{Shaded}
\begin{Highlighting}[]
\CommentTok{# compute estimated standard errors for correlation}
\NormalTok{se.rhohat =}\StringTok{ }\NormalTok{(}\DecValTok{1}\OperatorTok{-}\NormalTok{rhohat.vals}\OperatorTok{^}\DecValTok{2}\NormalTok{)}\OperatorTok{/}\KeywordTok{sqrt}\NormalTok{(nobs)}
\NormalTok{se.rhohat}
\end{Highlighting}
\end{Shaded}

\begin{verbatim}
## VBLTX,FMAGX  VBLTX,SBUX  FMAGX,SBUX 
##     0.08328     0.08336     0.06860
\end{verbatim}

\begin{Shaded}
\begin{Highlighting}[]
\KeywordTok{cbind}\NormalTok{(rhohat.vals,se.rhohat)}
\end{Highlighting}
\end{Shaded}

\begin{verbatim}
##             rhohat.vals se.rhohat
## VBLTX,FMAGX     0.06376   0.08328
## VBLTX,SBUX     -0.05614   0.08336
## FMAGX,SBUX      0.42384   0.06860
\end{verbatim}

\begin{Shaded}
\begin{Highlighting}[]
\CommentTok{# compute approx 95% confidence intervals}
\NormalTok{rho.lower =}\StringTok{ }\NormalTok{rhohat.vals }\OperatorTok{-}\StringTok{ }\DecValTok{2}\OperatorTok{*}\NormalTok{se.rhohat}
\NormalTok{rho.upper =}\StringTok{ }\NormalTok{rhohat.vals }\OperatorTok{+}\StringTok{ }\DecValTok{2}\OperatorTok{*}\NormalTok{se.rhohat}
\KeywordTok{cbind}\NormalTok{(rho.lower,rho.upper)}
\end{Highlighting}
\end{Shaded}

\begin{verbatim}
##             rho.lower rho.upper
## VBLTX,FMAGX   -0.1028    0.2303
## VBLTX,SBUX    -0.2229    0.1106
## FMAGX,SBUX     0.2866    0.5610
\end{verbatim}

The variance and SD are estimated more precisely. The mean is not
estimated precisely. the cor of VBLTX is not very precise. the cor
between FMAGX and SBUX is precise. The 95\% for mean of SBUX and FMAGX
have negative and positive values and it's relatively wide, so it's not
good estimate The 95\% for SD and var is not very wide, thus it give us
more precise estimate The third line in the correkation estimate is
narrow.

2 d)

\begin{Shaded}
\begin{Highlighting}[]
\NormalTok{Value.at.Risk =}\StringTok{ }\ControlFlowTok{function}\NormalTok{(x,}\DataTypeTok{p=}\FloatTok{0.05}\NormalTok{,}\DataTypeTok{w=}\DecValTok{100000}\NormalTok{) \{}
\NormalTok{    x =}\StringTok{ }\KeywordTok{as.matrix}\NormalTok{(x)}
\NormalTok{    q =}\StringTok{ }\KeywordTok{apply}\NormalTok{(x, }\DecValTok{2}\NormalTok{, mean) }\OperatorTok{+}\StringTok{ }\KeywordTok{apply}\NormalTok{(x, }\DecValTok{2}\NormalTok{, sd)}\OperatorTok{*}\KeywordTok{qnorm}\NormalTok{(p)}
\NormalTok{    VaR =}\StringTok{ }\NormalTok{(}\KeywordTok{exp}\NormalTok{(q) }\OperatorTok{-}\StringTok{ }\DecValTok{1}\NormalTok{)}\OperatorTok{*}\NormalTok{w}
\NormalTok{    VaR}
\NormalTok{\}}

\KeywordTok{Value.at.Risk}\NormalTok{(ret.mat,}\DataTypeTok{p=}\FloatTok{0.05}\NormalTok{,}\DataTypeTok{w=}\DecValTok{100000}\NormalTok{)}
\end{Highlighting}
\end{Shaded}

\begin{verbatim}
##  VBLTX  FMAGX   SBUX 
##  -3720  -9838 -16895
\end{verbatim}

\begin{Shaded}
\begin{Highlighting}[]
\KeywordTok{Value.at.Risk}\NormalTok{(ret.mat,}\DataTypeTok{p=}\FloatTok{0.01}\NormalTok{,}\DataTypeTok{w=}\DecValTok{100000}\NormalTok{)}
\end{Highlighting}
\end{Shaded}

\begin{verbatim}
##  VBLTX  FMAGX   SBUX 
##  -5429 -13692 -23389
\end{verbatim}

As we can see, the largest VaR is for SBUX( makes sense). VBLTX has the
lowest VaR value


\end{document}
